\documentclass[10pt]{article}
\usepackage{polski}
\usepackage[utf8]{inputenc}
\usepackage{amsmath}
\usepackage{amsfonts}
\usepackage{tabto,lipsum}
\usepackage{calc}
\usepackage{xcolor}
\usepackage{graphicx}
\graphicspath{ {./img/} }
\usepackage{shadowtext}
\usepackage{hyperref}
\hypersetup{%
  colorlinks=false,% hyperlinks will be black
  linkbordercolor=red,% hyperlink borders will be red
  pdfborderstyle={/S/U/W 1}% border style will be underline of width 1pt
}
\usepackage[margin=1.5cm]{geometry}
\usepackage{algpseudocode}
\usepackage{algorithm}
\usepackage{listings}

\PassOptionsToPackage{usenames,dvipsnames,svgnames}{xcolor}
\usepackage{tikz}
\usetikzlibrary{arrows,positioning,automata}

\linespread{1.3}

\title{Obliczenia Naukowe,\\Laboratorium, Lista 1}
\author{Jerzy Wroczyński}
\date{2020-10-25}

\begin{document}

\maketitle

\section{Zadanie 1.}

\subsection{Pod-zadanie 1.1.}

\subsubsection{Opis problemu}
Epsilonem maszynowym (\textit{macheps}) nazywamy najmniejszą liczbę w arytmetyce \textit{fl} większą od $0$ będącą odległością między $1$ a następną liczbą. Naszym zadaniem jest jej wyznaczenie.

\subsubsection{Rozwiązanie}
Zadanie dzieli się na kilka mniejszych pod-zadań wyznaczania tej wartości osobno dla arytmetyk \texttt{Float16}, \texttt{Float32} oraz \texttt{Float64}.
W języku \texttt{Julia} mamy dwa sposoby na uzyskanie takiej liczby:
\begin{itemize}
    \item metoda iteracyjna (kod źródłowy w pliku \texttt{ex-1.1.jl})
    \item wykonanie wbudowanej funkcji \texttt{eps}
\end{itemize}
W przypadku języka \texttt{C} musimy spojrzeć na predefiniowane globalne zmienne w pliku nagłówkowym \texttt{float.h}.

Zaobserwowane wartości:
\begin{itemize}
    \item \texttt{Float16}:
    \begin{itemize}
        \item metoda iteracyjna = $9.77 \cdot 10^{-4}$
        \item \texttt{eps(Float16)} = $9.77 \cdot 10^{-4}$
    \end{itemize}
    \item \texttt{Float32}:
    \begin{itemize}
        \item metoda iteracyjna = $1.1920929 \cdot 10^{-7}$
        \item \texttt{eps(Float32)} = $1.1920929 \cdot 10^{-7}$
        \item \texttt{float.h: \_\_FLT32\_EPSILON\_\_} = $1.1920928955078125 \cdot 10^{-7}$
    \end{itemize}
    \item \texttt{Float64}:
    \begin{itemize}
        \item metoda iteracyjna = $2.220446049250313 \cdot 10^{-16}$
        \item \texttt{eps(Float64)} = $2.220446049250313 \cdot 10^{-16}$
        \item \texttt{float.h: \_\_FLT64\_EPSILON\_\_} = $2.22044604925031308084726333618164062 \cdot 10^{-16}$
    \end{itemize}
\end{itemize}

Jak widać, we wszystkich przypadkach uzyskania żądanych stałych mamy te same wartości. Co oznacza, że nasza metoda iteracyjna jest poprawna.

\subsection{Pod-zadanie 1.2.}

\subsubsection{Opis problemu}
Liczbę maszynową $\eta$ nazywamy najmniejszą liczbę w arytmetyce \textit{fl} niebędącą zerem. Naszym zadaniem jest jej wyznaczenie.

\subsubsection{Rozwiązanie}
Tak jak w poprzednim pod-zadaniu, patrzymy na trzy różne arytmetyki \texttt{Float16}, \texttt{Float32} oraz \texttt{Float64}. Żeby uzyskać nasz wynik, wystarczy nieco zmodyfikować program użyty w poprzednim pod-zadaniu.

\begin{itemize}
    \item \texttt{Float16}
    \begin{itemize}
        \item metoda iteracyjna = $6 \cdot 10^{-8}$
        \item \texttt{nextfloat(fl(0.0))} = $6 \cdot 10^{-8}$
    \end{itemize}
    \item \texttt{Float32}
    \begin{itemize}
        \item metoda iteracyjna = $1 \cdot 10^{-45}$
        \item \texttt{nextfloat(fl(0.0))} = $1 \cdot 10^{-45}$
    \end{itemize}
    \item \texttt{Float64}
    \begin{itemize}
        \item metoda iteracyjna = $5 \cdot 10^{-324}$
        \item \texttt{nextfloat(fl(0.0))} = $5 \cdot 10^{-324}$
    \end{itemize}
\end{itemize}

Jak widać, wartości wyznaczone metodą iteracyjną nie różnią się od tych, wbudowanych w język \texttt{Julia} co oznacza, że metoda ta jest prawidłowa.

\subsection{Pod-zadanie 1.3.}
\textit{Jaki związek ma liczba \textit{macheps} z precyzją arytmetyki (oznaczaną na wykładzie przez $\epsilon$)?}

Obie te liczby mają taką samą wartość, jednakże nie opisują dokładnie tego samego. Liczba $\epsilon$ opisuje największy możliwy błąd względny w przypadku zaokrąglania liczby rzeczywistej. W zasadzie liczba \texttt{macheps} opisuje również taki największy możliwy błąd względny, jaki może się pojawić wokół liczby $1.0$.

\subsection{Pod-zadanie 1.4.}
\textit{Jak związek ma liczba $\eta$ z liczbą $\mathrm{MIN}_{sub}$ z wykładu?}

Obie te liczby opisują tę samą własność — najmniejsza liczba w danej arytmetyce \textit{fl}, która jest większa od $0$ (bez normalizacji tej liczby).

\subsection{Pod-zadanie 1.5.}
\textit{Co zwracają funkcje \texttt{floatmin(Float32)} oraz \texttt{floatmin(Float64)} i jaki jest związek zwracanych wartości z liczbą $\mathrm{MIN}_{nor}$ z wykładu?}

Wynik programu \texttt{ex-1.5.jl}:
\begin{itemize}
    \item \texttt{floatmin(Float32)} = $1.1754944 \cdot 10^{-38}$
    \item \texttt{floatmin(Float64)} = $2.2250738585072014 \cdot 10^{-308}$
\end{itemize}

Odpowiadające sobie wartości mają taki sam rząd wielkości i są prawie równe sobie. Opisują tę samą rzecz — najmniejsza liczba rzeczywista większa od zera, przy czym żądana wartość jest znormalizowana.

\subsection{Pod-zadanie 1.6.}

\subsubsection{Opis problemu}
Należy uzyskać wartość liczby granicznej $\mathrm{MAX}$ dla wszystkich typów zmiennopozycyjnych\\ \texttt{Float16}, \texttt{Float32} oraz \texttt{Float64}.

\subsubsection{Rozwiązanie}
Zadanie dzieli się na kilka mniejszych pod-zadań wyznaczania tej wartości osobno dla arytmetyk \texttt{Float16}, \texttt{Float32} oraz \texttt{Float64}.
W języku \texttt{Julia} mamy dwa sposoby na uzyskanie takiej liczby:
\begin{itemize}
    \item metoda iteracyjna (kod źródłowy w pliku \texttt{ex-1.6.jl})
    \item wykonanie wbudowanej funkcji \texttt{floatmax}
\end{itemize}

Zaobserwowane wartości:
\begin{itemize}
    \item \texttt{Float16}:
    \begin{itemize}
        \item metoda iteracyjna = $6.5 \cdot 10^{4}$
        \item \texttt{floatmax(Float16)} = $6.55 \cdot 10^{4}$
    \end{itemize}
    \item \texttt{Float32}:
    \begin{itemize}
        \item metoda iteracyjna = $3.4028235 \cdot 10^{38}$
        \item \texttt{eps(Float32)} = $3.4028235 \cdot 10^{38}$
        \item wartość podana na wykładzie: $3.4 \cdot 10^{38}$
    \end{itemize}
    \item \texttt{Float64}:
    \begin{itemize}
        \item metoda iteracyjna = $1.7976931348623157 \cdot 10^{308}$
        \item \texttt{eps(Float64)} = $1.7976931348623157 \cdot 10^{308}$
        \item wartość podana na wykładzie: $1.8 \cdot 10^{308}$
    \end{itemize}
\end{itemize}

Jak widać, wszystkie wartości się zgadzają, co oznacza, że zastosowana metoda iteracyjna jest poprawna.

\subsubsection{Opis metody iteracyjnej}
(kod źródłowy w pliku \texttt{ex-1.6.jl})
\begin{enumerate}
    \item Wypełniamy mantysę przy pomocy funkcji \texttt{sum\_}, która po kolei dodaje kolejne potęgi dwójki.
    \item Następnie mnożymy otrzymaną liczbę przez $2$ w pętli aż nie otrzymamy nieskończoności w rozumieniu danej arytmetyki (wykorzystujemy funkcję \texttt{isinf} do sprawdzania liczb)
    \item Bierzemy jako wynik liczbę poprzedzającą tę, która jest już nieskończonością w tym ciągu generowanym przez tę pętlę.
\end{enumerate}

\section{Zadanie 2.}

Wykonując polecenie \texttt{./ex-2.jl} otrzymujemy wyniki wyrażenia w arytmetykach \texttt{Float16},\\ \texttt{Float32} oraz \texttt{Float64}.

\begin{itemize}
    \item \texttt{Float16}:
    \begin{itemize}
        \item metoda Kahana = $9.77 \cdot 10^{-4}$
        \item \texttt{eps(Float16)} = $9.77 \cdot 10^{-4}$
    \end{itemize}
    \item \texttt{Float32}:
    \begin{itemize}
        \item metoda Kahana = $1.1920929 \cdot 10^{-7}$
        \item \texttt{eps(Float32)} = $1.1920929 \cdot 10^{-7}$
    \end{itemize}
    \item \texttt{Float64}:
    \begin{itemize}
        \item metoda Kahana = $2.220446049250313 \cdot 10^{-16}$
        \item \texttt{eps(Float64)} = $2.220446049250313 \cdot 10^{-16}$
    \end{itemize}
\end{itemize}

Metoda Kahana działa — wartości zwracane przez funkcję \texttt{kahans\_eps} są równe wartościom $\epsilon$ odpowiednich arytmetyk.

\section{Zadanie 3.}

\subsection{Pod-zadanie 3.1.}\label{z3.1.}

\subsubsection{Opis problemu}
Należy sprawdzić eksperymentalnie (w języku \textit{Julia}), że w arytmetyce \texttt{Float64} każdą liczbę $x \in [1;2]$ możemy zapisać w postaci

\begin{equation}\label{z3.1.wyr}
    x = 1 + k\cdot \delta
\end{equation}
gdzie $\delta = 2^{-52},~ k \in \{0,1,2,\dots, 2^{52}\}$.

\subsubsection{Rozwiązanie}
(kod źródłowy w pliku \texttt{ex-3.1.jl})

Program dołączony do tego zadania pokazuje przy pomocy ciągów jedynek i zer reprezentacje poszczególnych liczb w arytmetyce \texttt{Float64} w przedziale $[1;2]$. Zmienna \texttt{starting\_point} zmienia, od którego miejsca zacząć generować liczby z przedziału $[1;2]$.

\begin{itemize}
    \item \texttt{001111111111000000000000000000000000000000000000000000000000\textbf{0000}}
    \item \texttt{001111111111000000000000000000000000000000000000000000000000\textbf{0001}}
    \item \texttt{001111111111000000000000000000000000000000000000000000000000\textbf{0010}}
    \item \texttt{001111111111000000000000000000000000000000000000000000000000\textbf{0011}}
    \item \texttt{001111111111000000000000000000000000000000000000000000000000\textbf{0100}}
    \item \texttt{001111111111000000000000000000000000000000000000000000000000\textbf{0101}}
    \item \texttt{001111111111000000000000000000000000000000000000000000000000\textbf{0110}}
    \item \texttt{001111111111000000000000000000000000000000000000000000000000\textbf{0111}}
    \item \texttt{001111111111000000000000000000000000000000000000000000000000\textbf{1000}}
    \item \texttt{001111111111000000000000000000000000000000000000000000000000\textbf{1001}}
    \item \texttt{001111111111000000000000000000000000000000000000000000000000\textbf{1010}}
\end{itemize}

Widać po ciągach bitowych (należy skupić się nad końcem ciągów), że zachodzi odliczanie wszystkich liczb z tego przedziału po kolei. Można to zjawisko porównać do zwykłego liczenia liczb naturalnych w systemie binarnym.

\noindent To samo zjawisko zachodzi dla innych części tego przedziału.

Zaprezentowane dowody empiryczne skłaniają do wniosku, że wyrażenie \ref*{z3.1.wyr} jest prawdziwe.

\subsection{Pod-zadanie 3.2.}
(kod źródłowy w pliku \texttt{ex-3.2.jl})

Tym razem rozważamy przedział $\left[ \frac{1}{2}; 1 \right]$. Wprowadzamy modyfikacje do programu z pod-zadania \ref*{z3.1.}:
\begin{itemize}
    \item zmieniamy początkową wartość \texttt{x} na \texttt{0.5}
    \item zmieniamy wartość $\delta$ na $2^{-53}$
\end{itemize}
i zauważamy, że ponownie mamy to samo zjawisko — generowanie kolejnych liczb z zadanego przedziału.

\begin{itemize}
    \item \texttt{0011111111100000000000000000000001000000000000000000000000001010}
    \item \texttt{0011111111100000000000000000000001000000000000000000000000001011}
    \item \texttt{0011111111100000000000000000000001000000000000000000000000001100}
    \item \texttt{0011111111100000000000000000000001000000000000000000000000001101}
    \item \texttt{0011111111100000000000000000000001000000000000000000000000001110}
    \item \texttt{0011111111100000000000000000000001000000000000000000000000001111}
\end{itemize}

Na powyższych wynikach jest pokazana po prostu inna część tego przedziału — zachodzi takie samo zjawisko.

Czyli liczba $x \in \left[ \frac{1}{2}; 1 \right]$ może zostać zapisana w następujący sposób
$$
x = 0.5 + k \cdot \delta
$$
gdzie $k \in \{ 0,1,2,\dots, 2^{53} \},~ \delta = 2^{-53}$.

\subsubsection{Pod-zadanie 3.3.}
(kod źródłowy w pliku \texttt{ex-3.3.jl})

W przypadku przedziału $[2;4]$ mamy dokładnie taką samą sytuację, przy czym zaczynamy oczywiście od $2$, kiedy $\delta = 2^{-51}$.

\section{Zadanie 4.}

\subsection{Opis problemu}

($fl = \texttt{Float64}$)

\begin{enumerate}
    \item Należy wyznaczyć liczbę $1 < x < 2$, taką, że $fl(x \cdot fl(1/x)) \neq 1$.
    \item Należy znaleźć najmniejszą taką liczbę.
\end{enumerate}

\subsection{Rozwiązanie}

Wynik programu \texttt{ex-4.jl} spełnia oba podpunkty zadania. Program zaczyna od liczby $1 + \delta$ ($\delta$ taka sama jak w zadaniu \ref*{z3.1.}) i przegląda wszystkie liczby po kolei aż nie zostanie spełniony warunek \texttt{fl( x * fl(1/x)) != 1}.

Liczba znaleziona przez program wynosi $1.000001417355248$.

\section{Zadanie 5.}

\subsection{Opis problemu}
Należy napisać program liczący iloczyn skalarny dwóch wektorów przy pomocy czterech metod w dwóch arytmetykach (\texttt{Float32} oraz \texttt{Float64}).

$$
\begin{aligned}
    x &= [2.718281828, -3.141592654, 1.414213562, 0.5772156649, 0.3010299957]\\
    y &= [1486.2497, 878366.9879, -22.37492, 4773714.647, 0.000185049]
\end{aligned}
$$

\subsection{Rozwiązanie}

(kod źródłowy w pliku \texttt{ex-5.jl})

Aby uruchomić program, należy wykonać polecenie \texttt{./ex-5.jl} — wyświetlą się instrukcje mówiące, jakie argumenty są przyjmowane przez program.

Wyniki:
\begin{itemize}
    \item \texttt{Float32}
    \begin{itemize}
        \item metoda \textit{„w przód”} $= -0.2499443$
        \item metoda \textit{„w tył”} $= -0.2043457$
        \item od \textbf{największego} do \textit{najmniejszego} $= -0.25$
        \item od \textit{najmniejszego} do \textbf{największego} $= -0.25$
    \end{itemize}
    \item \texttt{Float64}
    \begin{itemize}
        \item metoda \textit{„w przód”} $= 1.0251881368296672e-10$
        \item metoda \textit{„w tył”} $= -1.5643308870494366e-10$
        \item od \textbf{największego} do \textit{najmniejszego} $= 0$
        \item od \textit{najmniejszego} do \textbf{największego} $= 0$
    \end{itemize}
\end{itemize}

Faktyczna wartość wynosi $-1.00657107000000 \cdot 10^{-11}$.

\textit{Każda z metod, nieważne w jakiej arytmetyce, zwraca nieprawidłową wartość. Jedynie w arytmetyce \texttt{Float64} mamy nieco zbliżone do rzeczywistości wyniki — jednak nadal nie są one satysfakcjonujące.}

\section{Zadanie 6.}

\subsection{Opis problemu}
Należy policzyć wartości funkcji
$$
\begin{aligned}
    f(x) &= \sqrt{x^2 + 1} - 1\\
    g(x) &= \frac{x^2}{\sqrt{x^2 + 1} + 1}
\end{aligned}
$$
dla argumentów $x = 8^{-1}, 8^{-2}, 8^{-3}, \dots$ oraz porównać wyniki.

\subsection{Rozwiązanie}

Wyniki:
\noindent
\makebox[\linewidth]{%
  \begin{minipage}[t]{\dimexpr0.5\linewidth-.2pt}
    \vspace{-\baselineskip}
    $$
    \begin{aligned}
        f(8^{-1}) &= 0.0077822185373186414\\
        g(8^{-1}) &= 0.0077822185373187065\\
        f(8^{-2}) &= 0.00012206286282867573\\
        g(8^{-2}) &= 0.00012206286282875901\\
        f(8^{-3}) &= 1.9073468138230965 \cdot 10^{-6}\\
        g(8^{-3}) &= 1.907346813826566 \cdot 10^{-6}\\
        f(8^{-4}) &= 2.9802321943606103 \cdot 10^{-8}\\
        g(8^{-4}) &= 2.9802321943606116 \cdot 10^{-8}\\
        f(8^{-5}) &= 4.656612873077393 \cdot 10^{-10}\\
        g(8^{-5}) &= 4.6566128719931904 \cdot 10^{-10}\\
        f(8^{-6}) &= 7.275957614183426 \cdot 10^{-12}\\
        g(8^{-6}) &= 7.275957614156956 \cdot 10^{-12}\\
    \end{aligned}
    $$
  \end{minipage}%
%   \vrule
  \begin{minipage}[t]{\dimexpr0.5\linewidth-.2pt}
    \vspace{-\baselineskip}
    $$
    \begin{aligned}
        f(8^{-7}) &= 1.1368683772161603 \cdot 10^{-13}\\
        g(8^{-7}) &= 1.1368683772160957 \cdot 10^{-13}\\
        f(8^{-8}) &= 1.7763568394002505 \cdot 10^{-15}\\
        g(8^{-8}) &= 1.7763568394002489 \cdot 10^{-15}\\
        f(8^{-9}) &= 0.0\\
        g(8^{-9}) &= 2.7755575615628914e{-17}\\
        f(8^{-10}) &= 0.0\\
        g(8^{-10}) &= 4.336808689942018e{-19}\\
        f(8^{-11}) &= 0.0\\
        g(8^{-11}) &= 6.776263578034403e{-21}\\
        f(8^{-12}) &= 0.0\\
        g(8^{-12}) &= 1.0587911840678754e{-22}\\
    \end{aligned}
    $$
  \end{minipage}
}

\noindent Uruchamiając program \texttt{ex-6.jl} widzimy, że faktycznie wartości się różnią.

\noindent Różnica oczywiście wynika z tego, że funkcje prezentują dwa różne sposoby policzenia tego samego. I właśnie to, że funkcja $g$ wykonuje większą liczbę działań, jest tutaj kluczowe. Większa liczba działań oznacza większą liczbę popełnianych błędów w zaokrągleniach.

Jednakże możemy zauważyć jeszcze jedną rzecz. Otóż przy wartości $8^{-9}$ argumentu funkcja $f$ zwraca już wartość $0$, kiedy funkcja $g$ nadal zwraca pewne wartości \textit{relatywnie podobne} do tych dla większych argumentów.

Trudno jest powiedzieć, która funkcja jest bardziej \textit{wiarygodna}, jednakże moim zdaniem funkcja $f$ wygrywa prostszą strukturą działań, jest ich po prostu mniej, czyli mniej możliwości popełnionych błędów.

\section{Zadanie 7.}

\subsection{Opis problemu}

Należy obliczyć wartość pochodnej funkcji $f(x) = \sin x + \cos 3x$ przy pomocy wzoru na przybliżoną pochodną
$$
f'(x_0) \approx \tilde{f'}(x_0) = \frac{f(x_0 + h) - f(x_0)}{h}
$$
oraz porównać z faktyczną wartością pochodnej.
\begin{itemize}
    \item $h = 2^{-n}$, dla $n = 0, 1, 2, \dots, 54$
    \item błędy mierzymy przy pomocy: $|f'(x_0) - \tilde{f'(x_0)}|$
\end{itemize}

\subsection{Rozwiązanie}

(kod źródłowy w pliku \texttt{ex-7.jl})

Faktyczną wartość pochodnej możemy obliczyć przy pomocy funkcji $f'(x) = \cos x - 3\sin 3x$.

\noindent Żeby zauważyć moment, w którym zmniejszanie wartości $h$ nie poprawia przybliżenia wartości pochodnej, patrzymy na wartość błędów (odchylenia przybliżenia od faktycznej wartości pochodnej).

\noindent W programie zakres dla liczby $n$ został ograniczony do przedziału $[-25; -32]$ (potęga $2$ w liczbie $h$) w celu ułatwienia odczytania, w którym miejscu zachodzi to zjawisko.

\begin{itemize}
    \item \begin{verbatim}
n = -26
h                         = 1.4901161193847656e-8
f(x + h) - f(x)           = 1.7425766385414931e-9
\tilde{f'}(x)             = 0.11694233864545822
f'(x)                     = 0.11694228168853815
|f'(x) - \tilde{f'}(x)|   = 5.6956920069239914e-8
    \end{verbatim}
    \item \begin{verbatim}
n = -27
h                         = 7.450580596923828e-9
f(x + h) - f(x)           = 8.712881527372929e-10
\tilde{f’}(x)             = 0.11694231629371643
f’(x)                     = 0.11694228168853815
|f’(x) - \tilde{f’}(x)|   = 3.460517827846843e-8
    \end{verbatim}
    \item \begin{verbatim}
n = -28
h                         = 3.725290298461914e-9
f(x + h) - f(x)           = 4.35643965346344e-10
\tilde{f’}(x)             = 0.11694228649139404
f’(x)                     = 0.11694228168853815
|f’(x) - \tilde{f’}(x)|   = 4.802855890773117e-9
    \end{verbatim}
    \item \begin{verbatim}
n = -29
h                         = 1.862645149230957e-9
f(x + h) - f(x)           = 2.1782187165086953e-10
\tilde{f’}(x)             = 0.11694222688674927
f’(x)                     = 0.11694228168853815
|f’(x) - \tilde{f’}(x)|   = 5.480178888461751e-8
    \end{verbatim}
    \item \begin{verbatim}
n = -30
h                         = 9.313225746154785e-10
f(x + h) - f(x)           = 1.0891088031428353e-10
\tilde{f’}(x)             = 0.11694216728210449
f’(x)                     = 0.11694228168853815
|f’(x) - \tilde{f’}(x)|   = 1.1440643366000813e-7
    \end{verbatim}
\end{itemize}

\noindent Najniższą wartość błędu otrzymujemy dla $n = -28$, dalsze wartości są tylko coraz większe.

\noindent Moment, w którym opisywane zjawisko zachodzi rząd wielkości liczb w liczniku i w mianowniku ułamka (będącego we wzorze aproksymacji pochodnej) jest bardzo niski ($-10$). Możliwe, że dochodzimy do granic możliwości dokładnego określenia, jaki powinien być wynik tego ułamka przez to, że te liczby są tak małe.

W przypadku kiedy przyjmujemy $x = 1 + h$ zamiast $x=1$ moment, w którym zachodzi to zjawisko jest przesunięty do momentu $n = -30$.

\end{document}
