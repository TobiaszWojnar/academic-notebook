\documentclass[14pt]{article}
\usepackage{polski}
\usepackage[utf8]{inputenc}
\usepackage{amsmath}
\usepackage{amsfonts}
\usepackage{tabto,lipsum}
\usepackage{xcolor}
\usepackage{graphicx}
\graphicspath{ {./img/} }
\usepackage{shadowtext}
\usepackage{hyperref}
\hypersetup{%
  colorlinks=false,% hyperlinks will be black
  linkbordercolor=red,% hyperlink borders will be red
  pdfborderstyle={/S/U/W 1}% border style will be underline of width 1pt
}
\usepackage[margin=3cm]{geometry}
\usepackage{algpseudocode}
\usepackage{algorithm}

\PassOptionsToPackage{usenames,dvipsnames,svgnames}{xcolor}
\usepackage{tikz}
\usetikzlibrary{arrows,positioning,automata}

\linespread{1.3}

\title{Lista 6}
\author{Zadanie 4}
\date{------------}

\begin{document}

\maketitle

\section{Problem}

Jedziemy samochodem palącym $1$ litr paliwa na $1$ km. Samochód ma górne ograniczenie paliwa, które może przewieźć wynoszące $W$. Na drodze są porozstawiane stacje paliwowe, gdzie $w_i$ to cena za $1$ litr paliwa na stacji $i$.
Należy znaleźć jak najtańszą drogę do stacji końcowej $n$.

\section{Concept}

Rozważamy drogę, po której jedzie samochód jako DAG (directed acyclic graph). Wierzchołkami w tym grafie są stacje. Krawędzie pomiędzy wierzchołkami tworzymy tylko wtedy kiedy pozwala na to zasięg samochodu. Jednakże, w takim modelu nie rozważamy sytuacji kiedy tankujemy więcej paliwa niż potrzebujemy do dojechania do następnej stacji, a może być sytuacja, że zatankowanie wcześniej większej ilości paliwa dałoby tańszą podróż. Musimy zatem rozszerzyć wierzchołki reprezentujące stacje o dodatkową informację — ilość pozostałego paliwa w baku samochodu.

Zatem rozbijamy wierzchołki reprezentujące poszczególne stacje na grupy wierzchołków oznaczanych przez parę $(k,i)$ gdzie $k$ to indeks stacji, a $i$ to liczba litrów paliwa pozostałego w baku po dojechaniu do tej stacji.

\section{Rozwiązanie}

Budujemy odpowiedni DAG $G=(V,E,f)$ gdzie $V$ to wierzchołki grafu, $E$ to krawędzie grafu a $f:V\times V \to \mathbb{R}$ to funkcja wag krawędzi.
Mamy wierzchołki $(k,i)$, gdzie $k$ to indeks stacji, a $i$ to liczba litrów paliwa „nadmiarowego” (ile litrów paliwa zostało po dojechaniu do $k$tej stacji) gdzie $i\in \{0,\dots,W\}$

Ustalmy ciąg $s_k$ określający kilometr stacji $k$ (odległość stacji $k$ od stacji startowej).
Wówczas odległość między stacjami $a$, $b$ wynosi $d_{a\to b} = s_b - s_a$.

Układ krawędzi i wierzchołków musi uwzględniać parametr $W$ (zasięg samochodu).
Zaczynamy od wierzchołka źródłowego $(1,0)$ który reprezentuje stację, na której samochód rozpoczyna podróż. Dla pozostałych stacji budujemy wierzchołki $(k,i)$ dla $i\in\{0,\dots,W\}$, $k\in\{2,\dots,n\}$.
Łączymy wierzchołki krawędziami uwzględniając zasięg oraz ile litrów paliwa „nadmiarowego” przewozi samochód. Czyli dla każdego wierzchołka $(k,i)$ rozważamy stacje o wierzchołkach $(l,j)$ takich, że $d_{k\to l} \le W - j$. Oczywiście nie łączymy wierzchołków dla tej samej stacji (czyli $\forall k~ \forall i,j~i\neq j:~ \big((k,i),(k,j)\big) \notin E$).

Funkcję wag krawędzi określamy jako $f\big((k,i),(l,j)\big)=w_k\cdot\big(d_{k\to l} - i + j \big)$. Wówczas za część przebytej drogi do następnej stacji możemy „zapłacić” paliwem kupionym na którejś z poprzednich stacji.

Stosujemy zmodyfikowany algorytm wyszukiwania najkrótszej drogi w DAG-u z zadania 1. Określamy wierzchołek startowy jako stację o indeksie $1$ — czyli w poniższym algorytmie $s = 0$.

\begin{algorithm}[H]
  \begin{algorithmic}[1]
    \State $d \gets$ matrix $n \times W$ filled with $\infty$
    \State $d[1][0] \gets 0$
    \State $r \gets$ matrix $n \times W$ filled with $[~]$
    \For{\textbf{each} $k$ in $\{2,\dots,n\}$}:
      \For{\textbf{each} $i$ in $\{0,\dots,W\}$}:
        \For{\textbf{each} $l$ in $\mathcal{N}(k)$}:
          \For{\textbf{each} $j$ in $\{0,\dots,W\}$}:
            \If{$d[k][i] > d[l][j] + f\big((k,i),(l,j)\big)$}:
              \State $d[k][i] \gets d[l][j] + f\big((k,i),(l,j)\big)$
              \State $r[k][i] \gets \texttt{concat}\Big(r[l][j], \big[ (k,i) \big]\Big)$
            \EndIf
          \EndFor
        \EndFor
      \EndFor
    \EndFor
  \end{algorithmic}
\end{algorithm}
Gdzie $\mathcal{N}$ to funkcja definiowana przez:
$\mathcal{N}(k) = \{l: \forall i,j\in\{0,\dots,W\}~ \forall \big((k,i),(l,j)\big) \in E\}.$

Tablica $r[n][0]$ reprezentuje ciąg wierzchołków określających kolejne stacje i liczbę litrów pozostałego paliwa na tych stacjach podczas najtańszej podróży do końca drogi. Liczba $d[n][0]$ reprezentuje koszt tej podróży.

W celu określenia na jakiej stacji ile bierzemy litrów paliwa musimy odpowiednio przetworzyć ciąg wynikowy $r[n][0]$: niech $R_p$ określa liczbę litrów paliwa pozostałą po dotarciu do stacji $p$tej (bierzemy stacje tylko z tego ciągu wynikowego z zachowaniem oryginalnych indeksów stacji).
Ciągiem określającym nasz ostateczny wynik będzie $S_n$, gdzie $S_p$ to liczba litrów paliwa, którą samochód musi zatankować na stacji $p$tej. Element tego ciągu obliczamy w następujący sposób: $S_p = d_{p\to p+1} + R_{p+1} - R_p$ przy czym $S_n = 0$.

\end{document}
