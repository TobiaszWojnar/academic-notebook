\documentclass[14pt]{article}
\usepackage{polski}
\usepackage[utf8]{inputenc}
\usepackage{amsmath}
\usepackage{amsfonts}
\usepackage{tabto,lipsum}
\usepackage{xcolor}
\usepackage{graphicx}
\graphicspath{ {./img/} }
\usepackage{shadowtext}
\usepackage{hyperref}
\hypersetup{%
  colorlinks=false,% hyperlinks will be black
  linkbordercolor=red,% hyperlink borders will be red
  pdfborderstyle={/S/U/W 1}% border style will be underline of width 1pt
}
\usepackage[margin=3cm]{geometry}
\usepackage{algpseudocode}
\usepackage{algorithm}

\PassOptionsToPackage{usenames,dvipsnames,svgnames}{xcolor}
\usepackage{tikz}
\usetikzlibrary{arrows,positioning,automata}

\linespread{1.3}

\title{Lista 6}
\author{Zadanie 4}
\date{------------}

\begin{document}

\maketitle

\section{Problem}

Jedziemy samochodem palącym $1$ litr paliwa na $1$ km. Samochód ma górne ograniczenie paliwa, które może przewieźć wynoszące $W$. Na drodze są porozstawiane stacje paliwowe, gdzie $w_i$ to cena za $1$ litr paliwa na stacji $i$.
Należy znaleźć jak najtańszą drogę do stacji końcowej $n$.

\section{Concept}

Reprezentujemy drogę z porozstawianymi stacjami jako DAG (directed acyclic graph) gdzie każda z krawędzi ma jakąś wagę. Odległość pomiędzy stacjami (wierzchołkami na grafie) możemy prosto obliczyć: $d_{i\to j} = i-j, i\le j$, gdzie $i$ oraz $j$ to indeksy stacji.
Wówczas wagi krawędzi można obliczać poprzez pomnożenie odległości między stacjami przez cenę paliwa w stacji początkowej dla tej krawędzi: $f\big((i,j)\in E\big) = d_{i\to j} \cdot w_i$. Jednakże, algorytm musi brać pod uwagę parametr $W$, będący ograniczeniem mówiącym, ile litrów może zostać wziętych przez samochód. Wystarczy wprowadzić „zakaz” tworzenia krawędzi pomiędzy stacjami, gdzie odległość jest większa niż $W$.

\section{Rozwiązanie}

Przygotowujemy odpowiedni graf. Sortowanie topologiczne nie jest potrzebne, bo trzymamy się oryginalnej kolejności stacji na drodze. Dalej, dla każdego wierzchołka generujemy krawędzie do wszystkich wierzchołków oddalonych o maksymalnie $W$.

Stosujemy algorytm wyszukiwania najkrótszej drogi w DAG-u z zadania 1. Określamy wierzchołek startowy jako stację o indeksie $0$ — czyli w poniższym algorytmie $s = 0$.

\begin{algorithm}[H]
  \begin{algorithmic}[1]
    \State $d \gets [\infty,\dots,\infty]$
    \State $d[s] \gets 0$
    \State $r \gets \big[~[~],~\dots,~[~]~\big]$
    \For{\textbf{each} $i$ in $\{1,\dots,n\}$}:
      \For{\textbf{each} $j$ in $\mathcal{N}(i)$}:
        \If{$d[i] > d[j] + f(j,i)$}:
          \State $d[i] \gets d[j] + f(j,i)$
          \State $r[i] \gets \texttt{concat}(r[j], [i])$
        \EndIf
      \EndFor
    \EndFor
  \end{algorithmic}
\end{algorithm}
Gdzie $\mathcal{N}$ to funkcja definiowana przez:
$\mathcal{N}(k) = \{j: \forall(v_k,v_j) \in E\}.$

Rozwiązaniem naszego problemu jest tablica $r[n]$ reprezentująca indeksy kolejnych stacji, które tworzą najtańszą ścieżkę oraz liczba $d[n]$ reprezentująca koszt tej podróży.

\end{document}
