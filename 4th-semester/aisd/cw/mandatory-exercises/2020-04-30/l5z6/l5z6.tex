\documentclass[14pt]{article}
\usepackage{polski}
\usepackage[utf8]{inputenc}
\usepackage{amsmath}
\usepackage{amsfonts}
\usepackage{tabto,lipsum}
\usepackage{xcolor}
\usepackage{shadowtext}
\usepackage{hyperref}
\hypersetup{%
  colorlinks=false,% hyperlinks will be black
  linkbordercolor=red,% hyperlink borders will be red
  pdfborderstyle={/S/U/W 1}% border style will be underline of width 1pt
}
\usepackage[margin=3cm]{geometry}

\linespread{1.3}

\title{Lista 5}
\author{Zadanie 6}
\date{------------}

\begin{document}

\maketitle

Głębokość danego węzła w RB-Tree nie można efektywnie utrzymywać jako dodatkowe pole węzła.
Należy zauważyć, że pole \texttt{depth} węzła $x$ opiera się na wartości tego samego pola węzła $x.\mathrm{parent}$. Dlatego, przy aktualizacji jednego węzła musimy dokonać aktualizacji wszystkich węzłów poniżej tego węzła przez co dochodzimy do złożoności obliczeniowej $O(n\cdot \log n)$.

\end{document}
