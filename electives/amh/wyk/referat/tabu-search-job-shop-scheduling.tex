\documentclass[14pt]{article}
% \usepackage{polski}
\usepackage[english]{babel}
\usepackage[utf8]{inputenc}
\usepackage{amsmath}
\usepackage{amsfonts}
\usepackage{xcolor}
\usepackage{graphicx}
\graphicspath{ {./img/} }
\usepackage{shadowtext}
\usepackage{hyperref}
\hypersetup{%
  colorlinks=false,% hyperlinks will be black
  linkbordercolor=red,% hyperlink borders will be red
  pdfborderstyle={/S/U/W 1}% border style will be underline of width 1pt
}
\usepackage[margin=2.5cm]{geometry}
\usepackage{algpseudocode}
\usepackage{algorithm}

%Import the natbib package and sets a bibliography  and citation styles
\usepackage[numbers]{natbib}
\bibliographystyle{plainnat}

\linespread{1.3}

\title{TabuSearch: Job-shop scheduling problem}
\author{Jerzy Wroczyński}
\date{2020-06-04}

\begin{document}

\maketitle

\section{Introduction}

The job shop scheduling problem is one of many theoretic scheduling problems. In a paper by \citet{amico-trubian} it was classified as $J || C_{\max}$ using the notation introduced by \citet{graham}. Letter $J$ represents „job shop scheduling problem”, two vertical lines with nothing in between mean no further job characteristics are given and $C_{\max}$ defines the optimization problem as minimizing the maximum completion time of all given jobs.

Of course, there are many different types of such problems e.g. there can be a predetermined quantity of machines e.g. only one machine, jobs can have certain characteristics e.g. each job has a \textit{fuzzy due date} etc. In this paper we will propose an algorithm for solving the problem classified as $J || C_{\max}$ and examine results of other algorithms used in other papers.

\hspace{2pt}

We are given following resources:
\begin{enumerate}
  \item a set $J$ of $n$ jobs to schedule,
  \item a set $O = \{1,\dots,N\}$ of $N$ atomic operations
  \item a set $M$ of $m$ machines.
\end{enumerate}

For each job $J_j$ there is a sequence of operations $O_{i,j} \in O$ and each of these operations has to be processed without interruption separately on a machine $\mu_{i,j} \in M$ for $d_{i,j}$ units of time.

For better understanding of a such schedule problem a visual aid of a Gantt chart can be used:
\begin{figure}[H]
  \centering
  \def\svgwidth{0.5\columnwidth}
  \input{img/example-gantt-chart.pdf_tex}
  \caption{Example Gantt chart}
  \label{example-gantt}
\end{figure}
which represents one feasible solution for scheduling three jobs (each consisting of three operations) on three machines. We can see now, there are some difficulties to overcome when dealing with such a problem. For example, in this solution machine $\mu_1$ runs idle for a very long time which can be an indicator on how good this solution is. Our main goal is to minimize the running time of the machine that completes its tasks last.

\section{Problem representation}

The Gantt chart very clearly shows the amount of time each task takes and for how long each machine runs idle. However, it is not a convenient way of representing this problem when it comes to applying TabuSearch.

\begin{figure}[H]
  \centering
  \def\svgwidth{0.5\columnwidth}
  \input{img/example-disjunctive-graph.pdf_tex}
  \caption{Example disjunctive graph}
  \label{example-disjunctive}
\end{figure}

Figure \ref{example-disjunctive} shows an example of a disjunctive graph that is commonly used when dealing with the job shop problem. Here, each atomic operation is represented as a node. The order of operations in each job is maintained using directed edges (black edges) drawn between appropriate operations (so they form a sequence of operations). These edges are called „conjunctive”. The rest of the edges (coloured edges) are called „disjunctive” and they represent all possible combinations of a job schedule. In this abstract form we can think of them as normal, undirected edges. When a schedule is being sought these edges need to become directed as to define order of execution on each of the given machines.

When searching for a solution the original order of operations in a job has to be maintained – direction of black edges cannot be altered. Defining the direction of the undirected edges will give us a certain schedule thus a solution to our problem. One way of directing these edges is to apply an „acyclic orientation” to each group of operations executed on the same machine. Then, having created such a DAG (Directed Acyclic Graph) the longest path (with respect to the cost function) defines the quality of this solution. In other words – the goal here is to minimize the cost of this longest path in our DAG.

\section{TabuSearch}

The heuristic we will be applying to this problem is TabuSearch. This heuristic algorithm is an extension to LocalSearch which searches through all neighbouring solutions of the currently chosen best solution. It chooses the next solution based on the cost function. TabuSearch introduces so called „Tabu” list that holds all forbidden „moves” that cannot be performed when generating a neighbourhood of the currently chosen solution. This reduces the amount of solutions to consider and thus gives more time for greater expansion in the global solution space.

To prevent from discarding some solutions due to the Tabu list we can add an aspiration criteria. For example if a given move has been put into Tabu list a long time ago, we can again use this move when generating a new neighbourhood.

\section{The proposed algorithm}

Before running the TabuSearch heuristic we need to define a way of finding the first feasible solution to start from.
Using the \textit{Shortest Processing Time} rule from \citet[section 2.1]{brandimarte} we can generate the first feasible solution. Each time there is more than one operation waiting to be processed by a given machine the operation with the lowest processing time is scheduled for running.

For the main algorithm we are going to use the one specified by \citet[section 2.]{amico-trubian}:

\begin{algorithm}[H]
  \begin{algorithmic}[1]
    \State $x^* \gets \text{find a feasible first solution}$
    \State $T \gets \emptyset$ $\triangleright$ the Tabu list
    \While{termination criteria not met}:
      \If{$T$ is full and there cannot be generated any more neighbouring solutions}: \label{ts-emptying-tabu}
        \State $T \gets \emptyset$
      \EndIf
      \State choose the best solution $\hat{x}$ from $\mathcal{N}(x^*, T)$ \label{ts-neighbourhood}
      \State put a move which leads from $\hat{x}$ to $x^*$ into $T$
      \If{$\hat{x}$ is better than $x^*$}:
        \State $x^* \gets \hat{x}$
      \EndIf
    \EndWhile
  \end{algorithmic}
  \caption{TabuSearch}
  \label{tabu-search}
\end{algorithm}

The function $\mathcal{N}$ used in line \ref{ts-neighbourhood} of our algorithm generates a neighbourhood of solutions with respect to the Tabu list. If no aspiration criteria is given or the aspiration criteria does not remove certain moves from the Tabu list, then the algorithm still works as it empties the Tabu list if no neighbourhood can be generated (condition in line \ref{ts-emptying-tabu}).

\subsection{Neighbourhood}

Many different approaches to the neighbourhood structure have been proposed over the years, but for our algorithm we will use the $N1$ neighbourhood introduced by \citet[section 3.3]{laarhoven}.
To generate a move we choose a tuple of vertices $v$ and $w$, such that:
\begin{enumerate}
  \item $v$ and $w$ are successive operation on some machine $k$
  \item $(v,w)$ is a critical edge meaning it is on the critical path (the longest path)
\end{enumerate}
and alter the order in which the $k$-th machine processes operations $v$ and $w$ through swapping both of these operations.

\subsection{Example}

Let us see an example of a small job shop problem and apply TabuSearch to it.

\begin{figure}[H]
  \centering
  \def\svgwidth{0.5\columnwidth}
  \input{img/example-ts-solution-1.pdf_tex}
  \caption{Exemple of a job shop scheduling problem}
  \label{ts-sol-1}
\end{figure}

Let us generate an initial solution:

\begin{figure}[H]
  \centering
  \def\svgwidth{0.5\columnwidth}
  \input{img/example-ts-solution-2.pdf_tex}
  \caption{Example job shop scheduling problem with initial solution applied}
  \label{ts-sol-2}
\end{figure}

The blue line in figure \ref{ts-sol-2} represents the longest path in this DAG.

\begin{figure}[H]
  \centering
  \def\svgwidth{0.5\columnwidth}
  \input{img/example-ts-solution-3.pdf_tex}
  \caption{Example job shop scheduling problem after reversing one of (originally disjunctive) edges on the critical path}
  \label{ts-sol-3}
\end{figure}

After reversing one edge as presented above in Figure \ref{ts-sol-3} the cost of the longest path is higher than the initial solution generated at the beginning.
Having performed a few iterations we find out that the first (initial) solution really was the best.

\section{Other algorithms used in literature}

We will compare the results of algorithms presented in papers by \citet{amico-trubian} and by \citet{pezzella}.

Of course, both of these approaches are entirely different. \citet{amico-trubian} have used their own List Scheduling method called \texttt{bidir} which creates two half-schedules that are generated starting from the beginning and from the end using the chosen priority rule.
\citet{pezzella} have also used a known method (proposed by \citet{adams}) for creating initial solutions called \textit{Shifting bottleneck} which


\bibliography{bib.bib}

\end{document}
